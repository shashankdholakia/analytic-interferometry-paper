% Define document class
\documentclass[modern]{aastex631}
\usepackage{showyourwork}
\usepackage{amsmath}

% Begin!
\begin{document}

% Title
\title{Analytic Interferometry of Stellar Surfaces with Spherical Harmonics}

% Author list
\author{Shashank Dholakia} \author{Benjamin J. S. Pope}

% Abstract with filler text
\begin{abstract}
\end{abstract}

% Main body with filleatmor text
\section{Introduction}
\label{sec:intro}
*Why is mapping the surfaces of stars important?

Recent advances in optical long-baseline interferometers have increasingly permitted the imaging and detection of features such as spots and faculae on the surfaces of individual stars. The longest baselines have so far permitted imaging of the surfaces of large, evolved stars with active chromospheres \citep{roettenbacher2016, martinez2021}. More recent advances could soon permit the detection of such features even on nearby main-sequence stars \citep{mourard2018, roettenbacher2022}.

Other techniques exist to probe the surfaces of stars outside the Sun that have been used on a wider variety of stars, such as Doppler imaging \citep{vogt1987}, light curve inversion \citep{harmon2000} (including using occultations from companion stars or exoplanets as in \citet{morris2017}). However, these methods can have limitations in their ability to definitively resolve features, demonstrated by \cite{roettenbacher2017}, whose comparative study revealed significant differences in spot maps obtained by the three methods of interferometry, Doppler imaging, and light curve inversion. 

These discrepancies can be understood from the information theory of each method. * understood as ill-posed problems with multiple solutions * Null spaces of each method are described in starry papers * often resolved with use of regularizers

Interferometry remains the only direct method for imaging the surfaces of stars which is possible using current technology. Nevertheless, interferometric imaging of stellar surfaces is itself an ill-posed problem due to the incomplete baseline coverage of the interferometer, which leads to significant gaps in the Fourier domain (or $uv$ plane). 

In Sec.~\ref{sec:maps}, we show that if a star's surface is described using spherical harmonics, there exists a linear operation with closed-form expressions to produce interferometric observables. In Sec.~\ref{sec:rotsynthesis}, we introduce the concept of `stellar rotation synthesis' as applied to the interferometry of resolved rotating bodies and describe the information content of such data. In  Sec.~\ref{sec:harmonix}, we describe the implementation of this model in the open-source code package \texttt{harmonix} developed in the \texttt{Jax} framework. Lastly, we summarize the methods and the paper in Sec.~\ref{sec:discussions}.
\section{Interferometry in the spherical harmonic basis}
\label{sec:maps}

Suppose that we have a spherical star whose intensity map, projected into a 2D at a specific viewing orientation, is defined as $I(x,y)$. Finding the total brightness of the star at that orientation is a sum over the visible surface of the star: 

\begin{equation}
F = \iint I(x,y) dx dy
\end{equation}
 
Finding the intensity map of a star given the disk-integrated brightness of a body as it rotates finds an closed-form solution in e.g. \citep{cowan2013} (and implemented in the open-source \texttt{starry} package by \citep{starry2019}) that, among other things, permits a description of the information content of photometric data. This is done first by representing the surface map of a star using the real spherical harmonics. If the 3d surface of a star is represented using spherical harmonic coefficients $\mathbf{y}$, then we can write the intensity map as:
 
\begin{equation}
    I(x,y) = \mathbf{\tilde{y}}^\top(x,y) \ \mathbf{R} \ \mathbf{y}
\end{equation}

where $\mathbf{\tilde{y}}^\top(x,y)$ is the spherical harmonic basis, $\mathbf{R}$ is the rotation matrix into the correct viewing orientation with the viewer at $+\infty$ along the z axis and $\mathbf{y}$ is the vector of spherical harmonic coefficients \citep{starry2019}. Because spherical harmonics form an orthonormal basis on a unit sphere, any map can be represented using a sufficiently high order expansion in the spherical harmonics. This makes it a natural choice to represent the surfaces of stars, planets and other approximately spherical bodies.

Here we show that it is possible to use the same description of the surface of a star in terms of spherical harmonics to compute analytic observables used in interferometry. Recall that interferometric observations record a quantity called the visibility. The van-Cittert Zernike theorem relates the intensity map of a star to its visibility using the Fourier transform:

\begin{equation}
V(u,v) = \iint I(x,y) e^{i(ux + vy)} dxdy
\end{equation}
where $V$ is the visibility at baseline (u,v). Using the spherical harmonic basis, the problem is linear, meaning we can write the integral as:

\begin{equation}
   V(u,v) = \iint \mathbf{\tilde{y}}^\top(x,y) \ e^{i(ux + vy)} \ dxdy \ \mathbf{R} \ \mathbf{y}
\end{equation}
where $\mathbf{R} \ \mathbf{y}$ are not dependant on x and y and can therefore be pulled out of the integral. 


\subsection{Analytic limb darkening}

\section{Stellar Rotation Synthesis}
\label{sec:rotsynthesis}
Null space description of interferometric data

Plot showing the null space of interferometry, DI and LI together
\section{Implementation in harmonix}
\label{sec:harmonix}
Plot showing the implementation agrees with brute force fourier transform of a mock star. 

Plot showing speed? show the jax part of the code off in some way, maybe do a fit of a mock star?
\section{Discussions}
\label{sec:discussions}

\bibliography{bib}

\end{document}
